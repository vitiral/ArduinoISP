%% LyX 2.0.2 created this file.  For more info, see http://www.lyx.org/.
%% Do not edit unless you really know what you are doing.
\documentclass[english]{article}
\usepackage[T1]{fontenc}
\usepackage[latin9]{inputenc}
\usepackage{listings}
\setlength{\parskip}{\smallskipamount}
\setlength{\parindent}{0pt}
\usepackage{babel}
\usepackage{array}
\usepackage[unicode=true]
 {hyperref}
\usepackage{breakurl}

\makeatletter

%%%%%%%%%%%%%%%%%%%%%%%%%%%%%% LyX specific LaTeX commands.
%% Because html converters don't know tabularnewline
\providecommand{\tabularnewline}{\\}

\makeatother

\begin{document}

\part*{ArduinoISP v 2}

Copyright (c) 2014 Garrett Berg, \href{http://garrett@cloudformdesign.com}{garrett@cloudformdesign.com}

Previous Copyright (c) 2008-2011 Randall Bohn as \href{http://ArduinoISP}{http://arduino.cc/en/Tutorial/ArduinoISP}

\textbf{}%
\begin{tabular}{|>{\centering}p{0.2\paperwidth}|>{\centering}p{0.4\paperwidth}|}
\hline 
Documentation: & \href{http://cloudformdesign.com/products/arduinoispv2}{http://cloudformdesign.com/products/arduinoispv2}\tabularnewline
\hline 
\hline 
Updated Version & \href{https://github.com/cloudformdesign/ArduinoISP/releases}{https://github.com/cloudformdesign/ArduinoISP/releases}\tabularnewline
\hline 
\hline 
BSD License & \href{http://www.opensource.org/licenses/bsd-license.php}{http://www.opensource.org/licenses/bsd-license.php}\tabularnewline
\hline 
\end{tabular}


\section*{Summary}

This sketch turns the Arduino into a AVRISP with the ability to communicate
via SofwareSerial afterwards.

It is the same as the original ArduinoISP, but with the added Talk
feature.


\section*{To use:}
\begin{itemize}
\item Open and load ISP\_talk onto your arduino, follow the directions \href{http://arduino.cc/en/Tutorial/ArduinoISP||http://arduino.cc/en/Tutorial/ArduinoISP}{here}.
Your programming board should now be connected and programmed.
\item To program your target device select the correct board and set the
programmer to \textit{<Arduino as ISP>}
\item Press \textit{<Cntrl+Shift+U>} to upload your sketch (instead of just
\textit{<Cntrl+U}>)
\item Follow directions below to take advantage of the talk feature.
\end{itemize}

\subsection*{New Talk Feature: }

After programming, the Arduino passes through all serial communication
(except the note below)

For most arduino boards, configure the software port with:

\begin{lstlisting}
SoftwareSerial SoftSerial(MOSI, MISO); // RX, TX
SoftSerial.begin(19200); // ISP_talk communicates at 19200 baud
\end{lstlisting}


IMPORTANT: ISP mode is triggered when the following is received from
the computer

\begin{lstlisting}
:: 0x30, 0x20, (delay 250ms +/- 5ms), 0x30, 0x20, (delay 250ms +/- 5ms), 0x30
\end{lstlisting}


Where {[}0x30, 0x20{]} are consecutive bits (within 800us of each
other)

Note that 0x30 == '0' and 0x20 == ' '

The targeted board can send any characters without triggering the
ISP


\subsection*{Examples}

Check out ISP\_receiver for a simple example of how to use this library

(it simply returns everything it receives)
\end{document}
